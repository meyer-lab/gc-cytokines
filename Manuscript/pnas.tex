\documentclass[9pt,twocolumn,twoside,lineno]{pnas-new}
% Use the lineno option to display guide line numbers if required.

\templatetype{pnasresearcharticle}

\title{$title$}

% Use letters for affiliations, numbers to show equal authorship (if applicable) and to indicate the corresponding author
$for(author)$
\author[$author.affilnum$]{$author.name$}
$endfor$

$for(affiliation)$
\affil[$affiliation.key$]{$affiliation.name$}
$endfor$

% Please give the surname of the lead author for the running footer
\leadauthor{Farhat} 

% Please add a significance statement to explain the relevance of your work
\significancestatement{$sigstate$}

% Please include corresponding author, author contribution and author declaration information
\authorcontributions{$contrib$}
\authordeclaration{$interests$}
\equalauthors{\textsuperscript{1}A.M.F. contributed equally to this work with A.C.W.}
\correspondingauthor{\textsuperscript{2}To whom correspondence should be addressed. E-mail: a\@asmlab.org}

% At least three keywords are required at submission. Please provide three to five keywords, separated by the pipe symbol.
\keywords{Keyword 1 $|$ Keyword 2 $|$ Keyword 3 $|$ ...} 

\begin{abstract}
$abstract$
\end{abstract}

\dates{This manuscript was compiled on \today}
\doi{\url{www.pnas.org/cgi/doi/10.1073/pnas.XXXXXXXXXX}}

\begin{document}

\maketitle
\thispagestyle{firststyle}
\ifthenelse{\boolean{shortarticle}}{\ifthenelse{\boolean{singlecolumn}}{\abscontentformatted}{\abscontent}}{}


$for(include-before)$
$include-before$

$endfor$

$body$

\subsection*{Format}

Many authors find it useful to organize their manuscripts with the following order of sections;  title, author line and affiliations, keywords, abstract, significance statement, introduction, results, discussion, materials and methods, acknowledgments, and references. Other orders and headings are permitted.

\subsection*{Manuscript Length}

A standard 6-page article is approximately 4,000 words, 50 references, and 4 medium-size graphical elements (i.e., figures and tables). The preferred length of articles remains at 6 pages, but PNAS will allow articles up to a maximum of 12 pages.

\begin{figure}%[tbhp]
\centering
\includegraphics[width=.8\linewidth]{frog}
\caption{Placeholder image of a frog with a long example legend to show justification setting.}
\label{fig:frog}
\end{figure}


\begin{SCfigure*}[\sidecaptionrelwidth][t]
\centering
\includegraphics[width=11.4cm,height=11.4cm]{frog}
\caption{This legend would be placed at the side of the figure, rather than below it.}\label{fig:side}
\end{SCfigure*}

\subsection*{Digital Figures}

EPS, high-resolution PDF, and PowerPoint are preferred formats for figures that will be used in the main manuscript. Authors may submit PRC or U3D files for 3D images; these must be accompanied by 2D representations in TIFF, EPS, or high-resolution PDF format. Color images must be in RGB (red, green, blue) mode. Include the font files for any text.

Images must be provided at final size, preferably 1 column width (8.7cm). Figures wider than 1 column should be sized to 11.4cm or 17.8cm wide. Numbers, letters, and symbols should be no smaller than 6 points (2mm) and no larger than 12 points (6mm) after reduction and must be consistent. 

Figures and tables should be labelled and referenced in the standard way using the \verb|\label{}| and \verb|\ref{}| commands.

Figure \ref{fig:frog} shows an example of how to insert a column-wide figure. To insert a figure wider than one column, please use the \verb|\begin{figure*}...\end{figure*}| environment. Figures wider than one column should be sized to 11.4 cm or 17.8 cm wide. Use \verb|\begin{SCfigure*}...\end{SCfigure*}| for a wide figure with side legends.

\matmethods{Please describe your materials and methods here. This can be more than one paragraph, and may contain subsections and equations as required. Authors should include a statement in the methods section describing how readers will be able to access the data in the paper. 

\subsection*{Subsection for Method}
Example text for subsection.
}

\showmatmethods{} % Display the Materials and Methods section

\acknow{$acknowledge$}

\showacknow{} % Display the acknowledgments section

% Bibliography
\bibliography{pnas-sample}

\end{document}